%!TEX root = ../dokumentation.tex

\chapter{Neo4j (Graph)} \label{ch:neo4j}
\chapterauthor{Felix Hoffmann, Leopold Fuchs, Stephan Auf der Landwehr, and Luca Schwarz}

Exercitation qui duis voluptate do esse aute. Minim deserunt ex minim sunt cupidatat est fugiat in pariatur ullamco. Enim esse voluptate nulla et sunt sint labore non ut eiusmod et. Deserunt laboris ullamco occaecat esse reprehenderit anim. Deserunt aute laboris tempor est occaecat duis in cupidatat.

\section{Introduction - Neo4j} \label{sec:introNeo4j}

Exercitation qui duis voluptate do esse aute. Minim deserunt ex minim sunt cupidatat est fugiat in pariatur ullamco. Enim esse voluptate nulla et sunt sint labore non ut eiusmod et. Deserunt laboris ullamco occaecat esse reprehenderit anim. Deserunt aute laboris tempor est occaecat duis in cupidatat.

% \begin{minted}
% [
% frame=lines,
% framesep=2mm,
% baselinestretch=1.2,
% fontsize=\footnotesize,
% linenos
% ]
% {python}
%     import numpy as np
        
%     def incmatrix(genl1,genl2):
%         m = len(genl1)
%         n = len(genl2)
%         M = None #to become the incidence matrix
%         VT = np.zeros((n*m,1), int)  #dummy variable
        
%         #compute the bitwise xor matrix
%         M1 = bitxormatrix(genl1)
%         M2 = np.triu(bitxormatrix(genl2),1) 
% \end{minted}
