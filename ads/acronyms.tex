%!TEX root = ../dokumentation.tex

\addchap{\langabkverz}
%nur verwendete Akronyme werden letztlich im Abkürzungsverzeichnis des Dokuments angezeigt
%Verwendung:
%		\ac{Abk.}   --> fügt die Abkürzung ein, beim ersten Aufruf wird zusätzlich automatisch die ausgeschriebene Version davor eingefügt bzw. in einer Fußnote (hierfür muss in header.tex \usepackage[printonlyused,footnote]{acronym} stehen) dargestellt
%		\acs{Abk.}   -->  fügt die Abkürzung ein
%		\acf{Abk.}   --> fügt die Abkürzung UND die Erklärung ein
%		\acl{Abk.}   --> fügt nur die Erklärung ein
%		\acp{Abk.}  --> gibt Plural aus (angefügtes 's'); das zusätzliche 'p' funktioniert auch bei obigen Befehlen
%	siehe auch: http://golatex.de/wiki/%5Cacronym

\begin{acronym}[XXXXXXXX]
    \doublespacing
    \setlength{\itemsep}{-\parsep}
    \acro{ACID}{Atomicity, Consistency, Isolation, Durability}
    \acro{API}{Application Programming Interface}
    \acro{JSON}{JavaScript Object Notation}
    \acro{BSON}{Binary JSON}
    \acro{XML}{Extensible Markup Language}
    \acro{YAML}{YAML Ain't Markup Language}
    \acro{XDCR}{Cross Datacenter Replication}
    \acro{RDBMS}{Relational Database Management System}
    \acro{CAP}{Consistency, Availability, Partition Tolerance}
    \acro{CRUD}{Create, Read, Update, Delete}
    \acro{CA}{Consistency and Availability}
    \acro{CP}{Consistency and Partition Tolerance}
    \acro{AP}{Availability and Partition Tolerance}
    \acro{NoSQL}{Not only SQL}
    \acro{PODC}{Symposium on Principles of Distributed Computing}
    \acro{PAAS}{Platform-as-a-Service}
    \acro{IMDG}{In-memory Data Grid}
    \acro{ORM}{Object-Relational Mapping}
    \acro{SQL}{Structured Query Language}
    \acro{IIT}{Indian Institute of Technology}
    \acro{UI}{User Interface}
    \acro{Bolt}[Bolt]{Bolt Protocol}
    \acro{JDK}{Java Development Kit}
\end{acronym}
