%!TEX root = ../dokumentation.tex

\chapter{Introduction} \label{ch:introduction}

The technology industry has recently undergone major changes and transformations due to advancements in areas like artificial intelligence, cloud computing, and cybersecurity. One important trend in the industry is the growing importance of data, which has led to the development of new technologies and tools to store, manage, and analyze data. \ac{NoSQL} databases can provide assistance here. Big data applications are excellent for \ac{NoSQL} databases since they are built to manage unstructured or semi-structured data. As business data creation rises, \ac{NoSQL} databases are being utilized more frequently, and this trend is expected to continue. 

Open-source \ac{NoSQL} databases, in particular, have gained popularity among developers and businesses alike because of their cost-effectiveness and community-driven development. However, choosing the right database for a specific task or application can be challenging as different databases offer different features and trade-offs.

The \ac{CAP} theorem, introduced by computer scientist Eric Brewer in 2000, is a theoretical framework that helps understand the trade-offs between consistency, availability, and partition tolerance in distributed systems. This theorem is essential to understanding the advantages and disadvantages of different \ac{NoSQL} databases and making informed decisions about which database to use for a particular task or application.

Therefore, this e-book aims to provide a comprehensive analysis of the advantages and disadvantages of current open-source \ac{NoSQL} databases with respect to the \ac{CAP} theorem and provide task/application-oriented recommendations for developers and businesses to make informed decisions about which database to use for their specific needs. This knowledge can help to meet the requirements regarding efficiency, scalability, and reliability of distributed systems, and ultimately lead to better products and services.

Considering the previously mentioned points, it is important to compare open-source \ac{NoSQL} databases. Here, the advantages and disadvantages of different \ac{NoSQL} databases are relevant. Likewise, the consideration of the databases with respect to a theoretical basis by the \ac{CAP} theorem of Brewer from the year 2000 is necessary. Based on the advantages and disadvantages, as well as the \ac{CAP} theorem, it is also interesting to develop task- or application-oriented recommendations. 

The e-book is limited to the 4 \ac{NoSQL} databases (Hazelcast, CouchbaseDB, Mongo DB, Neo4j) that are compared to each other and have different use cases. These databases are compared using the \ac{CAP} theorem and their differences with respect to \ac{SQL} databases are worked out. In addition, Elastic Search is considered an indexing and search tool for \ac{NoSQL} databases. Other types such as wide-column databases are not considered.

This book aims to provide an overview of key-value, document-oriented, and graph databases. Each type will be presented, outlining key factors and differences.

Furthermore, the functionalities are presented based on concrete databases. Hazelcast is featured as a key-value database. For document-oriented databases, MongoDB and Couchbase DB are introduced. Graph databases are represented with Neo4j. In addition to different databases, there is a section about Elastic Search, a search and indexing tool for \ac{NoSQL} databases.

Once a short history of a respective database has been given, it is shown how an instance of the database is set up, followed by the import and transformation of relational data, pointing out the differences to relational databases. Secondly, it is demonstrated how the data can then be accessed.

Thereafter, the fields of application, besides the advantages and disadvantages of the individual database are summarized, including possible deployment strategies regarding the \ac{CAP} Theorem.

Finally, a conclusion is drawn and a recommendation is made for which use cases the respective database might be used.

