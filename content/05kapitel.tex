%!TEX root = ../dokumentation.tex

\chapter{Neo4j (Graph)} \label{ch:neo4j}
\chapterauthor{Felix Hoffmann, Leopold Fuchs, Stephan Auf der Landwehr, and Luca Schwarz}

Exercitation qui duis voluptate do esse aute. Minim deserunt ex minim sunt cupidatat est fugiat in pariatur ullamco. Enim esse voluptate nulla et sunt sint labore non ut eiusmod et. Deserunt laboris ullamco occaecat esse reprehenderit anim. Deserunt aute laboris tempor est occaecat duis in cupidatat.

\section{Introduction} \label{sec:introductionNeo4j}

Exercitation qui duis voluptate do esse aute. Minim deserunt ex minim sunt cupidatat est fugiat in pariatur ullamco. Enim esse voluptate nulla et sunt sint labore non ut eiusmod et. Deserunt laboris ullamco occaecat esse reprehenderit anim. Deserunt aute laboris tempor est occaecat duis in cupidatat.

\section{Theory} \label{sec:theoryNeo4j}
% Authors: Stephan Auf der Landwehr, Luca Schwarz

Exercitation qui duis voluptate do esse aute. Minim deserunt ex minim sunt cupidatat est fugiat in pariatur ullamco. Enim esse voluptate nulla et sunt sint labore non ut eiusmod et. Deserunt laboris ullamco occaecat esse reprehenderit anim. Deserunt aute laboris tempor est occaecat duis in cupidatat.

\subsection{History} \label{subsec:historyNeo4j}

The history of Neo4j started in 2007 when the founder and CEO Chandra Rangan worked with other students on a Graph Database Project for the \ac{IIT}. The database was a success at the university, so the founders
decided to start a company. Neo4j has grown over the years and is very popular specifically among developers in India \parencite{historyneo4j}.

Nowadays, the company behind Neo4j is not a startup anymore. Many big companies, like North America's top 20 credit institutes, use this graph Database. In June 2022, Neo4j had more than 700 Employees around the globe.
The company recently released a platform where every developer can make suggestions about wished functionalities and even contribute
to the software written in Java. On the other hand, a closed source database can be bought, for example, by big companies \parencite{historyneo4j}. Also, a community Edition of the database, an open-source product, is available.

In the future, Neo4j wants to expand globally, focusing on India due to \enquote{a larger developer ecosystem} \parencite{historyneo4j}. However, in India and other regions like the US or Europe, the request 
for Neo4j is big. Therefore Neo4j is hiring country Managers to expand the presence in these countries too \parencite{historyneo4j}.

To sum it all up, Neo4j has experienced steady growth and is expanding around the globe. In the future, the continuation of growth can be assumed.

\subsection{Graph Model and functionality} \label{subsec:graphModelFunctionalityNeo4j}

% TODO: change reference -> reference chapter where you implemented an example
A graph Database works differently than a relational Database. Unlike a relational database, a graph database uses graphs to store data. Generally speaking, instead of using relations and linking them with foreign keys, a graph model uses edges to link and knots to store information. An example will be shown in the chapter \nameref{subsec:implementationPythonNeo4j}.

To understand a graph database and how it works, some basic graph theory is needed, as well as some problems, that exist within graph databases. First of all, a definition of a graph is required. A graph consists of nodes and edges. Nodes can be connected with edges, but can also stand alone. If every Node has an edge, it is called a connected graph. If any node is connected with any other node, the graph is called fully connected.
Furthermore, there are some specifications. An edge can be one directed or bi-directed. If a graph contains only one-directed edges, the graph is called \enquote{directed graph}. The edges can also contain weights, the resulting graph would then be called \enquote{weighted graph}. Weights and one directed edges are optional. If 2 nodes exists, that can't be endpoints, the graph is called disconnected, otherwise, it is a \enquote{connected graph} \parencite{graphBasics}.

With those basics, some problems within the graph theory, that are used in graph databases, can be discussed. The first problem is to find the shortest path between 2 nodes. For this purpose, the dijkstra algorithm exists. This algorithm finds the shortest path between 2 nodes but isn't quite fast. The algorithm starts at the starting node. Then in every iteration, it chooses the neighbor with the shortest way. This will be performed until all nodes have been reached.
The algorithm can adjust former results, therefore all nodes need to be considered. This makes the algorithm slow \parencite{dijkstra}.

The next problem is cycles within a graph. To detect cycles within an undirected a Breadth first search can be applied. Within a directed graph, a Depth First Traversal shows, if a cycle exists. Both algorithms are quite fast and therefore a cycle can be detected in a relatively short time \parencite{graphCircle} \parencite{cycle_directed}.

The last problem, that will be discussed is the \enquote{Königsberg Bridge Problem}. This problem asks the simple question, of whether a graph can be traversed so that each edge is visited only once, and if so, is there a point that can be start and endpoint. Such a way is called \enquote{eulerian way}. The
\enquote{Königsberg Bridge Problem}  itself describes the situation in Königsberg and can't be solved, because there is no such a way. t most 2 nodes with an odd number of edges. For a directed graph, it must also be considered how often and whether one can get back from an edge \parencite{koenigsberger}.

Neo4j works on the base of this theoretical background, but it also provides functionalities, that are not related to the graph theory. As mentioned in the chapter \nameref{subsec:historyNeo4j} Neo4j offers two editions, a community, and an enterprise edition. The difference between those two is that the enterprise edition also offers functionalities for clustering and scaling. Therefore the community edition works perfectly fine to set up a smaller
graph database for a personal use case where clustering is not required. Nevertheless, both editions offer a fully functional database, and every operation on the database can be performed. So customers are not forced to buy the enterprise edition to use the database to its fullest \parencite{Neo4jfeatures}.

Besides the 2 editions, there are also functionalities that are edition-independent. First, Neo4j offers a powerful query language called Cypher, explicitly designed for graph data. Cypher allows users to query the graph using natural language syntax and traverse the graph in real-time. Neo4j provides multiple indexing methods so that text-based searches can be performed.
It is possible for the enterprise version to scale horizontally, making the database highly performant \parencite{Neo4jfeatures}.

Neo4j is built to depict relations. Hence, the database is mostly used to display social networks or for fraud detection. The ability to detect fraud makes Neo4j very attractive for big enterprises because fraud is a common problem \parencite{Neo4jfeatures}. Neo4j can be integrated with various systems to use its benefits, which will be discussed in the chapter \nameref{subsec:advantagesDisadvantagesNeo4j}. Due to the possibility of integrating into other systems and the focus on displaying relations, Neo4j and other graph databases 
are used widely.

\subsection{Advantages and Disadvantages} \label{subsec:advantagesDisadvantagesNeo4j}

Exercitation qui duis voluptate do esse aute. Minim deserunt ex minim sunt cupidatat est fugiat in pariatur ullamco. Enim esse voluptate nulla et sunt sint labore non ut eiusmod et. Deserunt laboris ullamco occaecat esse reprehenderit anim. Deserunt aute laboris tempor est occaecat duis in cupidatat.

% TODO: Advantages related to RDBMS (GraphDB vs. RDBMS)
% TODO: Advantages related to Neo4j (Neo4j vs. other GraphDBs)

\section{Migration of a Relational Model} \label{sec:migrationRelationModelneo4j}
% Authors: Felix Hoffmann, Leopold Fuchs

Exercitation qui duis voluptate do esse aute. Minim deserunt ex minim sunt cupidatat est fugiat in pariatur ullamco. Enim esse voluptate nulla et sunt sint labore non ut eiusmod et. Deserunt laboris ullamco occaecat esse reprehenderit anim. Deserunt aute laboris tempor est occaecat duis in cupidatat.

\subsection{Installation} \label{subsec:installationNeo4j}

For personal use cases, Neo4j can be installed on a local machine. For this installation either the Desktop Applikation (e.g. `.exe`, AppImage) or the Linux package can be used. For a production environment, Neo4j can be deployed using Docker. For more advanced use cases using Kubernetes provides the option to scale a Neo4j database horizontally.

Following we will focus on deploying a single instance of Neo4j using a Docker container. 

\begin{figure}[H]
    \caption{Neo4j Docker Architecture} \label{fig:neo4jDockerArchitecture}
    \missingfigure{neo4j docker architecture}
\end{figure}

% TODO: Add numbers for ports

\autoref{fig:neo4jDockerArchitecture} shows our architecture we are going to use during this paper. As shown we deploy a single container which exposes two ports. The first port -- \dots -- exposed the \ac{ui} we can use to run Cypher queries. The second port -- \dots for this example -- exposes the \ac{bolt} which is used to communicate with the database.

\subsection{Implementation in Python} \label{subsec:implementationPythonNeo4j}

Exercitation qui duis voluptate do esse aute. Minim deserunt ex minim sunt cupidatat est fugiat in pariatur ullamco. Enim esse voluptate nulla et sunt sint labore non ut eiusmod et. Deserunt laboris ullamco occaecat esse reprehenderit anim. Deserunt aute laboris tempor est occaecat duis in cupidatat.

\subsection{Querying} \label{subsec:queryingNeo4j}

Exercitation qui duis voluptate do esse aute. Minim deserunt ex minim sunt cupidatat est fugiat in pariatur ullamco. Enim esse voluptate nulla et sunt sint labore non ut eiusmod et. Deserunt laboris ullamco occaecat esse reprehenderit anim. Deserunt aute laboris tempor est occaecat duis in cupidatat.

\section{Reflection} \label{sec:reflectionNeo4j}

Exercitation qui duis voluptate do esse aute. Minim deserunt ex minim sunt cupidatat est fugiat in pariatur ullamco. Enim esse voluptate nulla et sunt sint labore non ut eiusmod et. Deserunt laboris ullamco occaecat esse reprehenderit anim. Deserunt aute laboris tempor est occaecat duis in cupidatat.

\subsection{CAP Theorem} \label{subsec:capTheoremNeo4j}

Exercitation qui duis voluptate do esse aute. Minim deserunt ex minim sunt cupidatat est fugiat in pariatur ullamco. Enim esse voluptate nulla et sunt sint labore non ut eiusmod et. Deserunt laboris ullamco occaecat esse reprehenderit anim. Deserunt aute laboris tempor est occaecat duis in cupidatat.

\subsection{Conclusion} \label{subsec:conclusionNeo4j}

Exercitation qui duis voluptate do esse aute. Minim deserunt ex minim sunt cupidatat est fugiat in pariatur ullamco. Enim esse voluptate nulla et sunt sint labore non ut eiusmod et. Deserunt laboris ullamco occaecat esse reprehenderit anim. Deserunt aute laboris tempor est occaecat duis in cupidatat.
