%!TEX root = ../dokumentation.tex

\chapter{Neo4j (Graph)} \label{ch:neo4j}
\chapterauthor{Felix Hoffmann, Leopold Fuchs, Stephan Auf der Landwehr, and Luca Schwarz}

Exercitation qui duis voluptate do esse aute. Minim deserunt ex minim sunt cupidatat est fugiat in pariatur ullamco. Enim esse voluptate nulla et sunt sint labore non ut eiusmod et. Deserunt laboris ullamco occaecat esse reprehenderit anim. Deserunt aute laboris tempor est occaecat duis in cupidatat.

\section{Introduction} \label{sec:introductionNeo4j}

Exercitation qui duis voluptate do esse aute. Minim deserunt ex minim sunt cupidatat est fugiat in pariatur ullamco. Enim esse voluptate nulla et sunt sint labore non ut eiusmod et. Deserunt laboris ullamco occaecat esse reprehenderit anim. Deserunt aute laboris tempor est occaecat duis in cupidatat.

\section{Theory} \label{sec:theoryNeo4j}
% Authors: Stephan Auf der Landwehr, Luca Schwarz

Exercitation qui duis voluptate do esse aute. Minim deserunt ex minim sunt cupidatat est fugiat in pariatur ullamco. Enim esse voluptate nulla et sunt sint labore non ut eiusmod et. Deserunt laboris ullamco occaecat esse reprehenderit anim. Deserunt aute laboris tempor est occaecat duis in cupidatat.

\subsection{History} \label{subsec:historyNeo4j}

The history of Neo4j started in 2007 when the founder and CEO Chandra Rangan worked with other students on a Graph Database Project for the \ac{IIT}. The database was a success at the university, so the founders
decided to start a company. Neo4j has grown over the years and is very popular specifically among developers in India \parencite{historyneo4j}.

Nowadays, the company behind Neo4j is not a startup anymore. Many big companies, like North America's top 20 credit institutes, use this graph Database. In June 2022, Neo4j had more than 700 Employees around the globe.
The company recently released a platform where every developer can make suggestions about wished functionalities and even contribute
to the software written in Java. On the other hand, a closed source database can be bought, for example, by big companies \parencite{historyneo4j}. Also, a community Edition of the database, an open-source product, is available.

In the future, Neo4j wants to expand globally, focusing on India due to \enquote{a larger developer ecosystem} \parencite{historyneo4j}. However, in India and other regions like the US or Europe, the request 
for Neo4j is big. Therefore Neo4j is hiring country Managers to expand the presence in these countries too \parencite{historyneo4j}.

To sum it all up, Neo4j has experienced steady growth and is expanding around the globe. In the future, the continuation of growth can be assumed.

\subsection{Graph Model and functionality} \label{subsec:graphModelFunctionalityNeo4j}

A graph Database works differently than a relational Database. Unlike a relational database, a graph database uses graphs to store data. Generally speaking, instead of using relations and linking them with foreign keys, a graph model uses edges to link and knots to store information.

% TODO: Example doesn't make sense -> update or change it

For example, to save: Tom, 29 years old, a graph with the Knot \enquote{Tom}, and a knot with \enquote{29}, linked with an edge, is needed. These edges are directed edges, so only one directed connections exist. A bidirectional edge can be represented by two unidirectional edges \parencite{graphmodelneo4j}.

Inserting new data into a graph database is relatively easy and fast. Knots can be easily added to the graph, and the according edges can be created. The result is a graph that can grow and become quite big. The fact that a graph database does not need a primary key ensures that the model stays manageable. 
If a one-to-one transformation from a relational database to a graph database were performed (including ids), the model would be relatively large \parencite{graphmodelneo4j}.

Deleting data in graph models is also as easy as inserting new data. In a relational database, rules are necessary for updates or deletions like cascading or setting null. This is unnecessary in a graph database because there cannot be empty knots. Knots always have content inside them or will be removed. The only thing that could happen is that a knot has no edge outgoing and incoming. In this case, the knot
does not need to be deleted because other relations (edges) can be created later on \parencite{graphmodelneo4j} \parencite{funcneo4j}.

As mentioned in the chapter \nameref{subsec:historyNeo4j} neo4j offers two editions, a community, and an enterprise edition. The difference between those two is that the enterprise edition also offers functionalities for clustering and scaling. Therefore the community edition works perfectly fine to set up a smaller
graph database for a personal use case where clustering is not required. Nevertheless, both editions offer a fully functional database, and every operation on the database can be performed. So customers are not forced to buy the enterprise edition to use the database to its fullest \parencite{Neo4jfeatures}.

Neo4j offers a lot of functionality. First, Neo4j offers a powerful query language called Cypher, explicitly designed for graph data. Cypher allows users to query the graph using natural language syntax and traverse the graph in real-time. Neo4J provides multiple indexing methods so that text-based searches can be performed.
It is possible for the enterprise version to scale horizontally, making the database highly performant \parencite{Neo4jfeatures}.

Neo4j is built to depict relations. Hence, the database is mostly used to display social networks or for fraud detection. The ability to detect fraud makes Neo4j very attractive for big enterprises because fraud is a common problem \parencite{Neo4jfeatures}. Neo4j can be integrated with various systems to use its benefits, which will be discussed in the chapter \nameref{subsec:advantagesDisadvantagesNeo4j}. Due to the possibility of integrating into other systems and the focus on displaying relations, Neo4j and other graph databases 
are used widely.

\subsection{Advantages and Disadvantages} \label{subsec:advantagesDisadvantagesNeo4j}

Exercitation qui duis voluptate do esse aute. Minim deserunt ex minim sunt cupidatat est fugiat in pariatur ullamco. Enim esse voluptate nulla et sunt sint labore non ut eiusmod et. Deserunt laboris ullamco occaecat esse reprehenderit anim. Deserunt aute laboris tempor est occaecat duis in cupidatat.

% TODO: Advantages related to RDBMS (GraphDB vs. RDBMS)
% TODO: Advantages related to Neo4j (Neo4j vs. other GraphDBs)

\section{Implementation} \label{sec:implementationNeo4j}
% Authors: Felix Hoffmann, Leopold Fuchs

Exercitation qui duis voluptate do esse aute. Minim deserunt ex minim sunt cupidatat est fugiat in pariatur ullamco. Enim esse voluptate nulla et sunt sint labore non ut eiusmod et. Deserunt laboris ullamco occaecat esse reprehenderit anim. Deserunt aute laboris tempor est occaecat duis in cupidatat.

\subsection{Requirements} \label{subsec:requirementsNeo4j}

Exercitation qui duis voluptate do esse aute. Minim deserunt ex minim sunt cupidatat est fugiat in pariatur ullamco. Enim esse voluptate nulla et sunt sint labore non ut eiusmod et. Deserunt laboris ullamco occaecat esse reprehenderit anim. Deserunt aute laboris tempor est occaecat duis in cupidatat.

\subsection{Installation} \label{subsec:installationNeo4j}

Exercitation qui duis voluptate do esse aute. Minim deserunt ex minim sunt cupidatat est fugiat in pariatur ullamco. Enim esse voluptate nulla et sunt sint labore non ut eiusmod et. Deserunt laboris ullamco occaecat esse reprehenderit anim. Deserunt aute laboris tempor est occaecat duis in cupidatat.

\subsection{Implementation in Python} \label{subsec:implementationPythonNeo4j}

Exercitation qui duis voluptate do esse aute. Minim deserunt ex minim sunt cupidatat est fugiat in pariatur ullamco. Enim esse voluptate nulla et sunt sint labore non ut eiusmod et. Deserunt laboris ullamco occaecat esse reprehenderit anim. Deserunt aute laboris tempor est occaecat duis in cupidatat.

\subsection{Querying} \label{subsec:queryingNeo4j}

Exercitation qui duis voluptate do esse aute. Minim deserunt ex minim sunt cupidatat est fugiat in pariatur ullamco. Enim esse voluptate nulla et sunt sint labore non ut eiusmod et. Deserunt laboris ullamco occaecat esse reprehenderit anim. Deserunt aute laboris tempor est occaecat duis in cupidatat.

\section{Reflection} \label{sec:reflectionNeo4j}

Exercitation qui duis voluptate do esse aute. Minim deserunt ex minim sunt cupidatat est fugiat in pariatur ullamco. Enim esse voluptate nulla et sunt sint labore non ut eiusmod et. Deserunt laboris ullamco occaecat esse reprehenderit anim. Deserunt aute laboris tempor est occaecat duis in cupidatat.

\subsection{CAP Theorem} \label{subsec:capTheoremNeo4j}

Exercitation qui duis voluptate do esse aute. Minim deserunt ex minim sunt cupidatat est fugiat in pariatur ullamco. Enim esse voluptate nulla et sunt sint labore non ut eiusmod et. Deserunt laboris ullamco occaecat esse reprehenderit anim. Deserunt aute laboris tempor est occaecat duis in cupidatat.

\subsection{Conclusion} \label{subsec:conclusionNeo4j}

Exercitation qui duis voluptate do esse aute. Minim deserunt ex minim sunt cupidatat est fugiat in pariatur ullamco. Enim esse voluptate nulla et sunt sint labore non ut eiusmod et. Deserunt laboris ullamco occaecat esse reprehenderit anim. Deserunt aute laboris tempor est occaecat duis in cupidatat.
